\documentclass{article}
\usepackage{graphicx} % Required for inserting images

\title{Informe de Código Cumpleaños.py}
\author{Sebastián Jiménez Rocha}
\date{Mayo 2025}

\begin{document}

\maketitle

\section{Introducción}
Este informe cumple con la finalidad de describir y explicar el origen y funcionalidad del código presente en el directorio. Este fue desarrollado por mi con la única finalidad de cumplir con la tarea dada. Se trata de un programa escrito en Python, cuyo único y sencillo objetivo es solicitar la fecha del nacimiento del usuario, e informársela posteriormente.

\section{Autor y origen}
El Algoritmo fue creado por mi, \textbf{Sebastián Jiménez Rocha}, para demostrar mis capacidades en la creación y diseño de códigos, mientras cumplía con uno de los requisitos para la entrega de esta tarea. Fue pensado y diseñado tras un arduo debate intelectual conmigo mismo, como la mejor posible entrega que podría dar, dado que por motivos de integridad académica, no quiero ni ver código python escrito por alguien que no sea yo.

\section{Propósito del programa}
El programa tiene un propósito muy específico, acotado y simple: Obtener del usuario su día, mes y año de nacimiento, para finalmente mostrar esa misma fecha en formato \textbf{dia/mes/año}. Su funcionalidad es trivial y su \textbf{valor casi nulo}, pero por el tamaño de 1KB y su importancia académica en mi nota, el espacio que ocupa lo vale.

\section{Conclusión}
Aunque el código es simple, puede ejecutarse sin problemas, y puede ser de utilidad en el caso que olvides tu cumpleaños justo en el intervalo de tiempo en el que el código se ejecuta después de dictar tus respuestas.


\end{document}